
%%%%%%%%%%%%%%%%%%%%%%%%%%%%%%%%%%%%
\chapter{Research Methodology}
\label{chap:Literature_Review}
%%%%%%%%%%%%%%%%%%%%%%%%%%%%%%%%%%%%

\comment{

The Methodology section should provide a clear explanation of the research approach.  You want your reader to agree that you carefully considered your method so that we can trust your results to be both insightful (\underline{mean something}) and credible (\underline{not subject to error}):
\begin{itemize}
    \item A clear description of the methodology, how it creates a scientific investigation and operates to collect meaningful data.
    \item A clear justification of \underline{why} you have chosen this particular approach.
    \item Information needed for a reader to understand \underline{how} you did it (can a reader \underline{reproduce} your work, and collect equally valid results? e.g. hardware/software used, configuration, number of trials, any procedures used, etc.)
    \item A description of any approaches taken to process collected data, e.g. metrics are used to combine data in a meaningful way - you should state any used explicitly, their utility, their suitability to your methodology and their limitations.  
\end{itemize}



As on can see in Table \ref{tab:Table_with_numbers} there are numbers involved. 

%%%%%%%%%%%%%%%%%%%%%%%%%%%%%%%%%%%%
% If you have more complex tables you can 
% get a corresponding LaTeX code here
% https://www.tablesgenerator.com 
%%%%%%%%%%%%%%%%%%%%%%%%%%%%%%%%%%%%
\begin{table}[h!]
\centering
 \begin{tabular}{|c | c | c | c|} 
 \hline
  Frame number & User 1 state & User 2 state & Resulting state \\ [0.5ex] 
 \hline
 \hline
  n & 0 & 0 & 1 \\ 
 \hline
  n+1 & 0 & 1 & 2\\
 \hline
  n+2 & 1 & 0 & 3 \\
 \hline
  n+3 & 1 & 1 & 4 \\
 \hline
\end{tabular}
\caption{\label{tab:Table_with_numbers}An example of a table.}
\label{table:example}
\end{table}

For example, if $x>0$ then we can write
\begin{equation}
\label{eq:sum}
\sigma =\int_{x=0}^{\infty} \frac{1}{x^2}dx \quad ,
\end{equation}
where $\sigma$ is the integral (see Equations \ref{eq:sum}).  

}

% 本节中主要包含:程序的实现细节,实验场景的选择和设计

This section would cover:
\begin{itemize}
    \item Algorithm detail and implementation.
    \item Experiment designaiton.
\end{itemize}

\subsection*{Environment}
\subsubsection{Hardware}

\indent All the content mentioned in this project was carried out on a ROG Zephyrus M16 laptop.

The CPU in the laptop is 11th Gen Intel Core i7-11800H @ 2.30GHz.

All experiments and simulations in this study were conducted without GPU acceleration.

\subsubsection{Software}

\indent System: Ubuntu 22.04 LTS installed on WSL2 of Windows 11 23H2 22631.2129

ROS2: ROS2 Humble

All programs involved in this experiment were written in Python within this software environment.

\section{Algorithm Detail and Implementation}

The algorithm can be described as: 
\begin{enumerate}
    \item Agents establish self-organised communication among themselves using STDMA.
    \item Use the information transmitted in the channel for planning, and subsequently send their plans at the designated times.
\end{enumerate}

In this section, we will sequentially detail the principles and methods of implementation for these two components.

\subsection{STDMA Implementation}
\label{chap:STDMA Implementation}

Like mentioned before in Section \ref{chap:STDMA Explained}, in STDMA, agents have synchronised clock, share one channel, determine empty slots in frames, and apply for the slots needed, then use their own occupied slot for broadcasting.

\subsubsection{Assumptions}


Before delving into the details of the algorithm, it's necessary to first mention the assumptions of this algorithm.

\begin{itemize}
    \item \textbf{Synchronised Clock}: Agents have synchronised clock.
    \item \textbf{Slots and Frames}: Continuous time is represented with frames of equal length, and each frame includes an equal number of discrete slots, with each slot having the same duration.
    \item \textbf{Uniform Frame Length}: Frame length is predetermined for all agents and the value is the same.
\end{itemize}


\subsubsection{Synchronised Clock and Shared Channel}

These features are implemented with topics in ROS2\footnotemark. \footnotetext{http://docs.ros.org/en/humble/Concepts/Basic/About-Topics.html}
\begin{itemize}
    \item \textbf{Clock}
    
    A ROS2 node is created as clock signal generator and publisher.

    At specified intervals, this node toggles an internal Boolean value and sends this Boolean value to the clock topic. This results in a square wave with a 50\% duty cycle in the clock topic, where the period is adjustable.

    Agent nodes subscribe the clock topic to get the synchronised clock.

    \item \textbf{Division of Slots and Frames}
    
    A pair of True and False signal in the clock topic is considered as one slot by the agents.

    The frame length is predetermined within the agents, and all agents have a consistent frame length parameter.
    
    At the end of each slot, the agent increments its internal slot counter. If the slot count reaches one frame, it clears the slot counter and increments the frame counter. The algorithm does not require the frames to be synchronised between agents, i.e., the starting slot of the frame can differ for various agents.
    
    \textbf{In this implementation, all agents require only one slot.}

    \textbf{ADD PICTURE HERE}

    \item \textbf{Shared Channel}
    
    In the original STDMA\cite{STDMA}, only one channel is used for communicating and broadcasting. 
    But in the implemention here, \textbf{the channel is seperated in two, one for slot occupation messages, and one for other common messages}.
    
    \textbf{In the STDMA part, only the channel for slot occupation message is used.}

    Both of the channels here are implemented with ROS2 topic: All agent nodes subscribes to channel topics and could publish message to the channel topics.    

    Please note that this modification (splitting the channel into two) is purely for convenience and \textbf{does not impact its functionality in any way}.
    This modification in the current implementation is equivalent to using just one channel.    

\end{itemize}


\subsubsection{Node Actions within One Slot}

% 上升沿是开始,下降沿是中间

Before understanding how agents allocate slots, it's essential to know how agent acts within a slot.

As mentioned above, a slot is divided into two equal parts: one half with the clock signal being True, and the other half with the clock signal being False.

Given the characteristics of ROS2, the node operates in the following manner: each time it receives a new clock signal (when the clock node's boolean value changes and broadcasts the new boolean value), it executes a corresponding callback function.

\begin{itemize}
    \item \textbf{True} Received from the Clock Topic: 
    
    The clock signal's \textbf{rising edge}, treated as the \textbf{start of a slot}.

    Agents \textbf{send their message} if this slot belongs to them. 

    To be specific, agent nodes publish their message to the topic, and whenever an agent node receives a new message from the topic, it pushs the new message into a buffer.

    
    \item \textbf{False} Received from the Clock Topic:
    
    The clock signal's \textbf{falling edge}, treated as the \textbf{end of a slot}.
    
    Agents use the received messages to \textbf{update the current slot allocation and its state machine}.

    To be specific, agent nodes pop everything from the message buffer on slot's end, then use the messages received in the past slot to take corresponding actions.

\end{itemize}

\textbf{ADD PICTURE HERE}

\subsubsection{Slot Allocation Scheme and the State Machine}

Agents \textbf{always} listen to the channel and updates the local slots' allocation record.

There are two situations of the slot occupation:
\begin{itemize}
    \item \textbf{Occupied}: Only one agent sent message during this slot.
    \item \textbf{Free}: No agent sent message or multiple agents collided within one slot.
\end{itemize}

\textbf{ADD PICTURE HERE}  %给状态机画转化图解

The agent controls its behaviour through a finite state machine. 
The various states of this machine represent the agent's status within the network, reflecting the four phases in the STDMA protocol.
\textbf{Through the gradual transition of states within the state machine, the agent could join the network.}

\textbf{There are four states in the state machine}:

\begin{enumerate}
    \item \textbf{Listening}
    
    % 节点尚未进入网络,仅监听信道中的信息
    Agents in this state \textbf{have not joined the network yet}, and they only \textit{listen} to the messages in the channel.

    % Transition Conditions
    \textbf{Transition Conditions}:
    
    
    The agent \textbf{always records the allocation of past slots}, no matter which state it is in.
    
    \begin{itemize}
        \item \textbf{Empty}: 
        If no agent sends a message within a slot, or if messages from multiple agents collide in a slot, then that slot is recorded as empty. The reason for recording a slot as empty when a collision occurs is that, whenever an agent detects a collision, it will abandon the collided slot and seek a new slot for itself. 
        \item \textbf{Occupied}: 
        When only one agent attempts to send a message within a slot, that slot is recorded as occupied.       
    \end{itemize}

    Everytime finished listening to an entire frame, the agent will attempt to leave the Listening state.
    
    \begin{itemize}
        \item $\rightarrow$ Entering: When the number of free slots $\geq$ 1, transit to 'Entering'. Agent would randomly choose a free slot for later use, note that this doesn't mean that this chosen slot is already secured by this agent.
        \item $\circlearrowleft$ Listening: When no free slots left, stay in 'Listening'. This means the channel has already reached its maximum capacity and cannot take any other agents in. 
    \end{itemize}

    \item \textbf{Entering}
    
    Agent in this state would \textbf{wait for the free slot it selected} during the Listening state to arrive and then \textbf{attempt to claim that slot}, i.e., agent \textbf{sends its ID to the slot occupation message channel in the middle of the selected slot}.

    \textbf{Transition Conditions}:

    \begin{itemize}
        \item $\rightarrow$ Checking: Agent transit to the 'Checking' state after the occupation attempt message send.
    \end{itemize}


    \item \textbf{Checking}
    
    \textbf{Everytime} an agent sends a message in the middle of its selected slot or its occupied slot, it enters this state. At the end of the slot, based on the number of messages received in that slot, it transitions to another state.

    In a word, agents in this state is 'Checking' its right to use the slot.

    \textbf{Transition Conditions}:



    \begin{itemize}
        \item $\rightarrow$ In: If only itself sent a message (receiving only one message and the content is its own ID) in the selected or occupied slot, it means it has successfully occupied that slot or maintain occupation to the slot, i.e., 'In' the network.
        \item $\rightarrow$ Listening: If the agent doesn't successfully transition to the 'In' state after message broadcasting, it reverts to the 'Listening' state. 
        
        This condition encompasses multiple scenarios, including: 
        \begin{itemize}
            \item Collision: Multiple agents sent their messages in one slot. In this situation all senders reverse to 'Listening', which means that those attempting to claim the slot have failed, and those who previously occupied the slot relinquish it.
            \item Sending Failed: No message wasn't sent to the channel in this slot's duration.
        \end{itemize}
    \end{itemize}

    \textbf{Reminder about Collision}:

    Within the framework of this algorithm, the only potential collision between agents occurs when multiple agents happen to select the same free slot for occupation while 'Listening' and attempted to occupy that slot while 'Entering'.  

    In this situation, agent recognises the collision and reverse to 'Listening'.

    \item \textbf{In}
    
    Agents in this state are 'In' the network and not expected to drop off unless the agents decides to drop off (which could be simply done by stop sending regular message on their occupied slots). 

    \textbf{Transition Conditions}:

    \begin{itemize}
        \item $\rightarrow$ Checking: Like mentioned above, everytime an agent sent their message in the middle of their slot, it transits to 'Checking'.
    \end{itemize}

    \textbf{ADD PICTURE HERE} % 图示一下in和check和in的转化

    Agents in this state are expected to immediately transit back to 'In' after the checking at the end of the slot, because in theory no agent should try to send message in occupied slots.
    At times, collisions might occur between agents that haven't joined the network and those that have. However, such situations typically arise when one or more clock pulses are missed, resulting in a loss of synchronisation. Such situations should not happen or should be avoided as much as possible.

    The consequence of this situation is that the agent which has already joined the network is unexpectedly ejected from it. 
    Depending on the timing, various outcomes might ensue. For instance, if an operational agent is suddenly kicked out of the network, it may be unable to broadcast its information for at least one frame, potentially leading to accidents.

\end{enumerate}

% 写一个小总结?
\subsubsection{Summary}

Assumptions: Synchronised clock, predetermined frame length.

Functions: Self-organised single serial discrete channel sharing, allows agents to jumpin and leave.


\subsection{Path Planning Scheme and Implementation}


Now, we have a self-organised method of serial communication for multiple agents. 
Let's discuss how to implement decentralised collision-free path planning based on this basis.




\section{Experiment Design}
