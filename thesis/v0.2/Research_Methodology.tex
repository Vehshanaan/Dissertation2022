
%%%%%%%%%%%%%%%%%%%%%%%%%%%%%%%%%%%%
\chapter{Research Methodology}
\label{chap:Literature_Review}
%%%%%%%%%%%%%%%%%%%%%%%%%%%%%%%%%%%%

\comment{

The Methodology section should provide a clear explanation of the research approach.  You want your reader to agree that you carefully considered your method so that we can trust your results to be both insightful (\underline{mean something}) and credible (\underline{not subject to error}):
\begin{itemize}
    \item A clear description of the methodology, how it creates a scientific investigation and operates to collect meaningful data.
    \item A clear justification of \underline{why} you have chosen this particular approach.
    \item Information needed for a reader to understand \underline{how} you did it (can a reader \underline{reproduce} your work, and collect equally valid results? e.g. hardware/software used, configuration, number of trials, any procedures used, etc.)
    \item A description of any approaches taken to process collected data, e.g. metrics are used to combine data in a meaningful way - you should state any used explicitly, their utility, their suitability to your methodology and their limitations.  
\end{itemize}



As on can see in Table \ref{tab:Table_with_numbers} there are numbers involved. 

%%%%%%%%%%%%%%%%%%%%%%%%%%%%%%%%%%%%
% If you have more complex tables you can 
% get a corresponding LaTeX code here
% https://www.tablesgenerator.com 
%%%%%%%%%%%%%%%%%%%%%%%%%%%%%%%%%%%%
\begin{table}[h!]
\centering
 \begin{tabular}{|c | c | c | c|} 
 \hline
  Frame number & User 1 state & User 2 state & Resulting state \\ [0.5ex] 
 \hline
 \hline
  n & 0 & 0 & 1 \\ 
 \hline
  n+1 & 0 & 1 & 2\\
 \hline
  n+2 & 1 & 0 & 3 \\
 \hline
  n+3 & 1 & 1 & 4 \\
 \hline
\end{tabular}
\caption{\label{tab:Table_with_numbers}An example of a table.}
\label{table:example}
\end{table}

For example, if $x>0$ then we can write
\begin{equation}
\label{eq:sum}
\sigma =\int_{x=0}^{\infty} \frac{1}{x^2}dx \quad ,
\end{equation}
where $\sigma$ is the integral (see Equations \ref{eq:sum}).  

}

% 本节中主要包含:程序的实现细节,实验场景的选择和设计

This section would cover:
\begin{itemize}
    \item Algorithm detail and implementation.
    \item Experiment designaiton.
\end{itemize}

\subsection*{Environment}
\subsubsection{Hardware}

\indent All the content mentioned in this project was carried out on a ROG Zephyrus M16 laptop.

The CPU in the laptop is 11th Gen Intel Core i7-11800H @ 2.30GHz.

All experiments and simulations in this study were conducted without GPU acceleration.

\subsubsection{Software}

\indent System: Ubuntu 22.04 LTS installed on WSL2 of Windows 11 23H2 22631.2129

ROS2: ROS2 Humble

All programs involved in this experiment were written in Python within this software environment.

\section{Algorithm Detail and Implementation}

The algorithm can be described as: 
\begin{enumerate}
    \item Agents establish self-organised communication among themselves using STDMA.
    \item Use the information transmitted in the channel for planning, and subsequently send their plans at the designated times.
\end{enumerate}

In this section, we will sequentially detail the principles and methods of implementation for these two components.

\subsection{STDMA Implementation}
\label{chap:STDMA Implementation}

Like mentioned before in Section \ref{chap:STDMA Explained}, in STDMA, agents have synchronised clock, share one channel, determine empty slots in frames, and apply for the slots needed, then use their own occupied slot for broadcasting.


\begin{centering}
\textbf{ADD ASSUMPTIONS HERE!}
\end{centering}


\subsubsection{Synchronised Clock and Shared Channel}

These features are implemented with topics in ROS2\footnotemark. \footnotetext{http://docs.ros.org/en/humble/Concepts/Basic/About-Topics.html}
\begin{itemize}
    \item \textbf{Clock}
    
    A ROS2 node is created as clock signal generator and publisher.

    At specified intervals, this node toggles an internal Boolean value and sends this Boolean value to the clock topic. This results in a square wave with a 50\% duty cycle in the clock topic, where the period is adjustable.

    Agent nodes subscribe the clock topic to get the synchronised clock.

    \item \textbf{Division of Slots and Frames}
    
    A pair of True and False signal in the clock topic is considered as one slot by the agents.

    The frame length is predetermined within the agents, and all agents have a consistent frame length parameter.
    
    At the end of each slot, the agent increments its internal slot counter. If the slot count reaches one frame, it clears the slot counter and increments the frame counter. The algorithm does not require the frames to be synchronised between agents, i.e., the starting slot of the frame can differ for various agents.
    
    \textbf{ADD PICTURE HERE}

    \item \textbf{Shared Channel}
    
    In the original STDMA\cite{STDMA}, only one channel is used for communicating and broadcasting. 
    But in the implemention here, the channel is seperated in two, one for slot occupation messages, and one for other common messages. 

    Both of the channels here are implemented with ROS2 topic: All agent nodes subscribes to channel topics and could publish message to the channel topics.    

    Please note that this modification (splitting the channel into two) is purely for convenience and does not impact its functionality in any way.
    This modification in the current implementation is equivalent to using just one channel.    

\end{itemize}


\subsubsection{Node Actions within One Slot}

% 上升沿是开始,下降沿是中间

Before understanding how agents allocate slots, it's essential to know how agent acts within a slot.

As mentioned above, a slot is divided into two equal parts: one half with the clock signal being True, and the other half with the clock signal being False.

Given the characteristics of ROS2, the node operates in the following manner: each time it receives a new clock signal (when the clock node's boolean value changes and broadcasts the new boolean value), it executes a corresponding callback function.

\begin{itemize}
    \item \textbf{True} Received from the Clock Topic: 
    
    The clock signal's \textbf{rising edge}, treated as the \textbf{start of a slot}.

    Agents send their message if this slot belongs to them. 

    To be specific, agent nodes publish their message to the topic, and whenever an agent node receives a new message from the topic, it pushs the new message into a buffer.


    
    \item \textbf{False} Received from the Clock Topic:
    
    The clock signal's \textbf{falling edge}, treated as the \textbf{end of a slot}.
    
    Agents use the received messages to update the current allocation and its state machine.

    To be specific, agent nodes pop everything from the message buffer on slot's end, then use the messages received in the past slot to take corresponding actions.
\end{itemize}

\textbf{ADD PICTURE HERE}

\subsubsection{Slot Allocation Scheme and the State Machine}

The agent controls its behaviour through a finite state machine. 
The various states of this machine represent the agent's status within the network, reflecting the four phases in the STDMA protocol.
\textbf{Through the gradual transition of states within the state machine, the agent could join the network.}

\textbf{There are four states in the state machine}:

\begin{enumerate}
    \item \textbf{Listening}
    
    % 节点尚未进入网络,仅监听信道中的信息
    Agents in this state \textbf{have not joined the network yet}, and they only \textit{listen} to the messages in the channel.

    % Transition Conditions
    \textbf{Transition Conditions}:
    
    
    The agent \textbf{always records the allocation of past slots}.
    
    \begin{itemize}
        \item \textbf{Empty}: 
        If no agent sends a message within a slot, or if messages from multiple agents collide in a slot, then that slot is recorded as empty. The reason for recording a slot as empty when a collision occurs is that, whenever an agent detects a collision, it will abandon the collided slot and seek a new slot for itself. 
        \item \textbf{Occupied}: 
        When only one agent attempts to send a message within a slot, that slot is recorded as occupied.       
    \end{itemize}

    Everytime finished listening to an entire frame, the agent will attempt to leave the Listening state.
    
    \begin{itemize}
        \item $\rightarrow$ Entering: The number of free slots $\geq$ 1. Agent would randomly choose a free slot for later use, note that this doesn't mean that this chosen slot is already secured by this agent.
        \item $\circlearrowleft$ Listening: No free slots left. This means the channel has already reached its maximum capacity and cannot take any other agents in. 
    \end{itemize}

    \item \textbf{Entering}
    \item \textbf{Checking}
    \item \textbf{In}
\end{enumerate}




\section{Experiment Design}