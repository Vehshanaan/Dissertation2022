%%%%%%%%%%%%%%%%%%%%%%%%%%%%%%%%%%%%
\chapter{Algorithm}
\label{chap:Algorithm}
%%%%%%%%%%%%%%%%%%%%%%%%%%%%%%%%%%%%

\comment{

The Methodology section should provide a clear explanation of the research approach.  You want your reader to agree that you carefully considered your method so that we can trust your results to be both insightful (\underline{mean something}) and credible (\underline{not subject to error}):
\begin{itemize}
    \item A clear description of the methodology, how it creates a scientific investigation and operates to collect meaningful data.
    \item A clear justification of \underline{why} you have chosen this particular approach.
    \item Information needed for a reader to understand \underline{how} you did it (can a reader \underline{reproduce} your work, and collect equally valid results? e.g. hardware/software used, configuration, number of trials, any procedures used, etc.)
    \item A description of any approaches taken to process collected data, e.g. metrics are used to combine data in a meaningful way - you should state any used explicitly, their utility, their suitability to your methodology and their limitations.  
\end{itemize}



As on can see in Table \ref{tab:Table_with_numbers} there are numbers involved. 

%%%%%%%%%%%%%%%%%%%%%%%%%%%%%%%%%%%%
% If you have more complex tables you can 
% get a corresponding LaTeX code here
% https://www.tablesgenerator.com 
%%%%%%%%%%%%%%%%%%%%%%%%%%%%%%%%%%%%
\begin{table}[h!]
\centering
 \begin{tabular}{|c | c | c | c|} 
 \hline
  Frame number & User 1 state & User 2 state & Resulting state \\ [0.5ex] 
 \hline
 \hline
  n & 0 & 0 & 1 \\ 
 \hline
  n+1 & 0 & 1 & 2\\
 \hline
  n+2 & 1 & 0 & 3 \\
 \hline
  n+3 & 1 & 1 & 4 \\
 \hline
\end{tabular}
\caption{\label{tab:Table_with_numbers}An example of a table.}
\label{table:example}
\end{table}

For example, if $x>0$ then we can write
\begin{equation}
\label{eq:sum}
\sigma =\int_{x=0}^{\infty} \frac{1}{x^2}dx \quad ,
\end{equation}
where $\sigma$ is the integral (see Equations \ref{eq:sum}).  

}

This section provides detailed information on algorithm and its implementation.

\section*{Environment}

\begin{itemize}
    \item \textbf{Hardware:} ROG Zephyrus M16 Laptop
    \begin{itemize}
        \item CPU: 11th Gen Intel(R) Core(TM) 17-11800H @ 2.30GHz
        \item GPU: NVIDIA GeForce RTX 3060 Laptop GPU (unrelated to the experiment, information provided just for content completeness)
    \end{itemize}
    \item \textbf{Software:}
    \begin{itemize}
        \item OS: WSL2 (Ubuntu 22.04 LTS) in Windows 11 23H2
        \item Implementation Platform: ROS2 Humble, all codes written in python
    \end{itemize}
\end{itemize}

\section{Communication with STDMA}

In STDMA, agents share a single channel by autonomously determining the serial speaking order.
The method for determining the speaking order involves agents autonomously allocating the right to use free times within the channel.


\subsection{Synchronised Clock}

% STDMA假设agent之间拥有同步时钟。
\begin{quotation}
    \textbf{Assumption 1}: 
    Agents have synchronised clocks. 
\end{quotation}

% 在实际中,同步时钟一般用GPS实现。在本文的模拟中,用一个ROS2publisher和一个topic来实现。
In practice, the synchronised clock is typically implemented with GPS. 
In the implemented simulation of this paper, it's achieved using a ROS2 publisher and a topic.
% 一个专用的ros2节点定时翻转其成员bool值,并在每次翻转此值时将翻转后的bool值publish到时钟topic中,这样在时钟topic中来形成一个占空比为50%的方波时钟信号。
A dedicated ROS2 node periodically toggles its member boolean value and publishes the toggled value to the clock topic each time it's flipped. This creates a square wave clock signal with a 50\% duty cycle in the clock topic.
% 其他普通agent通过订阅时钟话题的方式来获得同步时钟信号。
Other standard agents obtain the synchronised clock signal by subscribing to the clock topic.

% 信道时间的离散化
\subsection{Discretisation of Channel Time}

% agent将一个完整的时钟信号周期视作一个slot,将若干slot视作一帧。
Agents consider a complete clock signal cycle as one slot and consider several slots as one frame.

\begin{quotation}
    \textbf{Assumption 2:} 
    The number of slots in a frame is predefined within the agents, and this parameter value is the same for all agents.
\end{quotation}

% 注意,不要求各agent的帧起始点相同,允许agent有不同的agent起始点offset
Note that it's not required for each agent to have the same frame starting point, 
i.e., agents are allowed to have different frame starting point offsets.

\subsection{State Machine for Channel Allocation}

% 使用STDMA的agent有四个阶段,每个阶段对应一个在网络中的状态。因此,agent加入网络的过程可以用状态机来管理,随着状态逐步转化,agent逐步加入网络并获得自己的槽
Agents using STDMA have four phases\cite{STDMA}, with each phase corresponding to a status of an agent joining the network. Therefore, the process of an agent joining the network can be managed using a state machine. As the state transitions progressively, the agent gradually joins the network and obtains its slot.

% 在解释状态机的运作前,有必要简述信道中对槽使用情况的表示和判断方式
Before explaining the operation of the state machine, it's necessary to briefly describe how the usage of slots in the channel is represented and determined: 
% 每当某agent占用的槽到来时,agent向信道中发送自己的id。
Agents publish their ID to the channel (which is a ROS2 topic) in the middle of the slot they occupy.
% agent保持对信道的监听以实时更新槽位分配情况记录。若一个槽中仅有一个agent发送信息,则此槽被此agent占有;若在一槽中无agent发送消息(未被使用)或有多个agent同时发送消息(碰撞),则此槽未被占有。
Agents keep listening to the channel (subscribe to the channel topic) to update their slot allocation records. If only one agent sends a message in a slot, that slot is occupied by that agent. If no agents send a message in a slot (i.e., it's unused) or if multiple agents send messages simultaneously (i.e., a collision), then the slot is considered unoccupied.
% Agent在每个槽结束时对本槽中接收到的消息数量计数,并进行上述判断。
At the end of each slot, agents count the number of messages received during that slot and make the aforementioned determination.

\textbf{ADD PIC HERE} % 绘制一个槽,槽中agent发信,槽末agent数数

% 本文中对此状态机的实现如下:
The implementation of this state machine in this paper is as follows:

\begin{enumerate}
    \item \textbf{Listening}
    
    % 在此状态下的agent如何如何如何

    % 此状态下的agent的目的是确定当前的信道槽位分配情况,此状态下的agent尚未进入网络
    In this state, the agent's objective is to determine the current channel slot allocation by 'Listening' to the channel. 
    Agents in this state have not yet joined the network.

    An agent's initial state is this state.

    % Transition Condition: 每次听完完整的一帧后,尝试离开此状态
    \textbf{Transition Condition}: After listening to a complete frame, the agent attempts to exit this state.
    \begin{itemize}
        % 当信道中有一个或多个未被使用的槽时,进入entering状态
        \item $\rightarrow$ Entering: There are one or more unoccupied slots in the frame.
        % 信道中无空闲槽时:呆在listening.
        \item $\circlearrowleft$ Listening: No free slot left in the frame. This indicates that the channel's capacity has reached its limit and the agent can only stay idle.
    \end{itemize}
    
    \item \textbf{Entering}
    
    % 在此状态下的agent的目的是如何,agent的状态是如何
    In this state, the agent attempts to occupy a free slot in order to try 'Entering' the network.

    % Transition Condition: 
    \textbf{Transition Condition}: 
    The agent tries to occupy an unused random slot. In the middle of this slot, the agent publishes its ID.

    \begin{itemize}
        \item $\rightarrow$ Checking: Agent transit to the 'Checking' state immediately after the occupation attempt message send.
    \end{itemize}


    \item \textbf{Checking}
    
    % 在此状态下的agent的目的是如何,agent的状态是如何
    % 此状态下的Agent刚刚在某一槽中发布了一条信息,需要根据槽内收到的消息的数量判断自己对此槽的所有权。
    In this state, the agent has just published a message in a certain slot and needs to determine its ownership of the slot based on the number of messages received within it.
    % 通过检查自己对此槽的所有权,变更自身状态。
    By 'Checking' its ownership of that slot, the agent transits its own state.

    % 注意,未进入网络和已进入网络的agent都会在发布消息后进入此状态。
    Note, both agents that haven't entered the network and those that have will enter this state after releasing a message.

    % Transition Condition:
    \textbf{Transition Condition}: Transit state based on the number of messages received in the slot where one has sent a message.
    \begin{itemize}
        % 如果只收到一条消息且来自自己:说明自己拥有此槽位,即在网络中
        \item $\rightarrow$ In: If only one message is received and it's from itself, it indicates that the agent has ownership of that slot, meaning it is 'In' the network.
        % 任何其他情况:自己不拥有此槽位,重新尝试获得槽位。
        \item $\rightarrow$ Listening: In any other situation: The agent does not have ownership of that slot and should try to secure a slot again.
        
        % 解释:所有其他情况包含:
        Explanation: All other situations include:
        \begin{itemize}
            % 没有收到自己的消息
            \item Didn't receive its own message: This indicates that the broadcast is not successful or receiving is not successful (e.g. hardware damaged or scheduled broadcasting prevented, etc.). In both situation, we don't want the agent to enter network, since its communicating function is not always fully functional. 
            % 多个节点尝试在同一槽内发送消息:
            \item Multiple agents sent messages within one slot (i.e. collision):
            
            % 碰撞情景1:多个正在进入网络的agent恰好选择了同一个槽位。此种碰撞在本算法框架下无法避免,但由于仅发生在未进入网络的agent之间,对其他有效通讯没有影响。
            Collision Scenario 1: Multiple agents trying to join the network happen to choose the same slot. Such collisions are inevitable within the framework of this algorithm, but since they only occur among agents that haven't entered the network, they don't affect other communications.

            % 碰撞情景2:已加入网络的agent与未加入网络的agent发生碰撞。此种碰撞在理论上是不应该发生的,因为未加入网络的节点应当只尝试获得未被使用的槽位。若agent错过了时钟脉冲,则可能会出现此种情况。若某agent失去时钟脉冲,则会导致其和所有其他agent失去同步,进而导致事故。这一情况要尽可能避免。
            Collision Scenario 2: A collision occurs between an agent that has joined the network and one that hasn't. Theoretically, this kind of collision shouldn't happen, as agents not yet in the network should only attempt to secure unused slots. If an agent misses a clock pulse, this situation might arise. If an agent misses the clock pulse, it will result in it falling out of sync with all other agents, potentially leading to accidents. This situation should be avoided as much as possible.

        \end{itemize}
    \end{itemize}

    \item \textbf{In}
    
    % 在此状态下的agent的目的是如何,agent的状态是如何
    % 在此状态下的agent已经进入网络,且应当始终在网络中,直到其通过停止定期发送消息来放弃其槽位。
    In this state, the agent is 'In' the network and should remain within the network, until it releases its slot by ceasing to send messages regularly.
    
    % Transition Condition:
    \textbf{Transition Condition}:
    \begin{itemize}
        % 每次发送消息后,检查自己对此槽的所有权
        \item $\rightarrow$ Checking: After message publishing, check its ownership of that slot.
    \end{itemize}

\end{enumerate}

\textbf{ADD PIC HERE} % 画一个状态机的图片 

\subsection{Summary}

% 实现了agent的自组织无碰撞信道分享。信道的时间被表示为包含指定数量槽的重复的帧,agent通过尝试获得帧中的一个槽的方式来加入网络。
The agent's self-organised collision-free channel sharing is implemented. The channel's time is represented as repetitive frames containing a specified number of slots. Agents join the network by attempting to secure a slot within a frame.

\section{Path Planning and Sharing}


% agent的移动模型, 计划的作用,分享的方式
\subsection{Necessary Explanation}

% 在解释agent的计划制定策略前,需要描述一下本文中agent的移动方式。
Before explaining the agent's planning strategy, it's necessary to state the details:

\subsubsection{Agent Mobility}

\begin{itemize}
    % 在每一个时间步(即一个stdma槽)中,agent可以选择待在原地,也可以选择移动到非对角的相邻grid。
    \item In each time step (i.e., an STDMA slot), the agent can choose to remain in its current position or move to an adjacent grid that's not diagonal.
    % 出于简化的考虑,agent的移动中没有扰动,也就是说,agent可以完全准确地按计划移动。
    \item Agents could execute their plan with complete accuracy. This is to simplify the mobility model.
\end{itemize}

\begin{quotation}
    \textbf{Assumption 3}: 
    % agent的移动不受外部扰动的干扰,且对自己运动的预测百分之百准确。
    The agent's movement is not affected by external disturbances, and its prediction of its own motion is completely accurate.
\end{quotation}

% 本文中所指的计划具体是什么
\subsubsection{Definition of Plan}
% 本文中,agent仅负责生成自身的计划。
In this paper, an agent is only responsible for generating its own plan.

% 计划的定义就是一串三维坐标点,其中每个坐标点都由一个二维坐标和一个时间点构成。
The definition of a plan is a sequence of 3D coordinate points, where each coordinate point is composed of a 2D coordinate and a time point, i.e. where and when.
% 由于agent mobility的约束,计划中相邻点的二维坐标部分必须为非对角的相邻点。此外,计划中的相邻点的时间点也必须是单调以1的步长递增。
Due to the constraints on agent mobility, the 2D coordinate part of adjacent points in the plan must be non-diagonal neighbors. Moreover, the time points of adjacent points in the plan must be monotonically increasing with a step length of 1.

% 计划一经发布,则agent必须按照已发布的计划部分移动。
Once a plan is published, the agent must move according to the published part of the plan.

% 计划的约束
\subsubsection{Constraints of the Plan}

% 计划的约束是:1. 不能与其他agent或地图中的障碍物发生碰撞。 2. 计划中相邻的点,空间上必须是非对角的相邻点,时间上后一点必须比前一点增加1。 3. 计划的第一个点必须在空间上与开始计划时刻的agent位置位于非对角的相邻点。
\begin{enumerate}
    \item Agent must not collide with other agents or obstacles on the map.
    \item Adjacent points in the plan must be non-diagonal neighbour points in 2D space, and the subsequent point in the plan must be one step further in time compared to the previous point.
    \item The first point of the plan must be a non-diagonal neighboring point in space to the agent's current position.
\end{enumerate}

\subsubsection{Collision Definition}
% 在本文中,以下两种情形被视为碰撞:
In this paper, the following two scenarios are considered as collisions:
\begin{itemize}
    % 两个或以上节点在同一时刻位于地图上的同一2D位置
    \item Two or more agents are located at the same 2D position on the map at the same time.
    % 两个节点交换位置。即,t时刻agentA位于位置a,agentB位于位置b,t+1时刻A位于b,而B位于A
    \item Two agents swapping their positions. That is, at time $t$, agent $A$ is at position $a$, and agent $B$ is at position $b$. At time $t+1$, agent $A$ is at position $b$, while $B$ is at position $a$.
\end{itemize}

\textbf{ADD PIC HERE} % 放一张描述碰撞的图

\subsubsection{Map}
The map is a grid world with obstacles, and:
% 本文中所使用的地图仅包含静止障碍物
\begin{quotation}
    \textbf{Assumption 4}:
    % 所有的障碍物都是静止的
    All obstacles in the map are stationary.
\end{quotation}

% 分享
\subsubsection{Plan Sharing}
% 分享计划的方式

% agent通过藉由STDMA协议所分享的信道通讯,分享的内容为自身未来若干步内的移动计划。
Agents communicate via the channel shared through the STDMA protocol, sharing their movement plans for the upcoming steps.

% 在本文的模拟中,为了实现的方便,分享的计划信息publish到一个单独的ROS2topic中。此信道的使用时间分配遵守由STDMA决定的顺序。
In the simulation of this paper, for the sake of implementation convenience, the shared plan is published to a separate ROS2 topic (not the topic used for STDMA). This channel's time slot allocation follows the order determined by STDMA.


% 计划的时机
\subsubsection{Time Window for Planning}

% 如前文所述,每个agent在属于自己的槽的中间发布信息。
As previously mentioned, each agent publishes their message (their plan and ID) in the middle of the slot that belongs to it.
% 基于这一特性,对于每个agent来说,存在这样一个定期出现的时间窗口:涉及计划制订的所有变量都是确定的,且自己的计划只要制订完成就可以立即发布。
Based on this rule, for each agent, there exists a regularly occurring time window 
in which all variables involved in plan-making are fixed, and the agent's own plan can be immediately published as soon as it is formulated.

% 这个时间窗口就是自己的槽的前半部分。在这段时间内,不会有其他agent发布新的计划,且此时段结束后就可以立即发布自身的计划。
This time window is the first half of the agent's own slot. During this time, no other agents will publish new plans, and as soon as this period ends, the agent can immediately publish its own plan.
% 由于此窗口是一个槽,这一窗口在每一帧中会出现一次
Since this time window is part of a slot, it will appear once in every frame.

\textbf{ADD PIC HERE}% 画图解释进行计划的时机

\subsubsection{Purpose of the Plan}
\begin{enumerate}
    \item Finding a continuous collision-free two-dimensional space that the agent can use.
    \item Help the agent to gradually approach and reach its goal..
\end{enumerate}

% 其中,第一个功能比第二个更重要。
The first purpose is more important than the second.

%本文中agent并不初始位于地图中的某个点,为agent指定的起点都位于地图的边缘且不重复,agent必须生成一个以其自身指定起点为开头的计划并publish才能进入地图中。
In this paper, agents do not initially occupy a specific point on the map. The starting points assigned to agents are all on the edge of the map and do not overlap. Agents must generate and publish a plan starting at their designated starting point in order to enter the map.
% 成功生成并发布一个计划就代表地图中存在足够的空间供此agent使用,故而此时允许agent进入地图。
Successfully generating and publishing a plan indicates that there is enough space in the map for this agent to use, so the agent is allowed to enter the map at this time.


\subsubsection{Path Planning Scheme}

% 在每个计划窗口开始时,计划函数被调用,用以生成接下来agent的移动计划。
At the start of each planning window, the planning function is called to generate the upcoming movement plan for the agent.

\textbf{Inputs of the function}:
\begin{itemize}
    % 地图,起点终点: 计划所需的基础信息
    \item map data, current position, goal
    
    These are basic information required for planning. 

    % 其他agent的计划
    \item plans of other agents
    
    Information for collision avoidance.

    % 是否是第一次进入地图? 
    \item first ever plan or not? (bool lean flag)
    
    % 指示是否是agent的第一个计划的bool值flag。若是,则计划起点须为分配给此agent的起始点
    A boolean value flag indicating whether it is the agent's first plan. If it is, the 2D coordinates of the first point of the plan must be the start assigned to this agent.
    
    % 所需计划之长度
    \item required plan length
    
    % 每次生成计划时要求生成的长度。
    The required length of the plan to be generated each time.

    % 由于agent每帧进行一次计划,而agent必须遵循其计划移动,所以每次生成的计划之长度必须至少等于一帧中slot的数量(即帧长度)。
    Since the agent plans once per frame, and the agent must follow its planned movement, the length of each generated plan must be at least equal to the number of slots in a frame (i.e., the frame length).

    % 如果agent的计划长度短于帧长度,会导致在帧末尾agent没有可使用的计划,其存在不为其他agent所知,而导向潜在的碰撞。
    If the agent's plan length is shorter than the frame length, it will result in the agent having no available plan near the end of the frame, and its presence will be unknown to other agents, leading to potential collisions.
    % 这一限制可以通过一些适当的技巧解决,如:agent之间对无计划时的移动有一些共识,进而可以用这一共识预测其他agent的行动,来允许计划长度短于两次计划之间的间隔。
    This limitation can be addressed through some appropriate techniques, such as having a consensus among agents on movement when there is no plan, which can then be used to predict the actions of other agents, allowing the plan length to be shorter than the interval between two planning windows.
    % horizon
    \item predicting horizon
    
    % 算法所允许预测的horizon上限
    The upper limit of the horizon length allowed for prediction in single planning window.
    % 若一条可能路径的长度达到horizon,则本次计划中不再继续扩展此路径。
    If the length of a potential path reaches the horizon, then this path will not be further extended in the current planning session.
    
    % 这一限制主要来自于运算速度。
    This limitation mainly arises from the computational speed, and:
    \begin{quotation}
        \textbf{Assumption 5}: 
        The predicting horizon upper limit is predefined within the agents, and this parameter value is the same for all agents.
    \end{quotation}
    % 固定的horizon是一种简化处理,统一的horizon有利于控制的简化。
    % Assumption n: agent有默认的预测horizon上限长度。

\end{itemize}