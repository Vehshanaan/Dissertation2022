
%%%%%%%%%%%%%%%%%%%%%%%%%%%%%%%%%%%%
\chapter{Research Methodology}
\label{chap:Literature_Review}
%%%%%%%%%%%%%%%%%%%%%%%%%%%%%%%%%%%%

\comment{

The Methodology section should provide a clear explanation of the research approach.  You want your reader to agree that you carefully considered your method so that we can trust your results to be both insightful (\underline{mean something}) and credible (\underline{not subject to error}):
\begin{itemize}
    \item A clear description of the methodology, how it creates a scientific investigation and operates to collect meaningful data.
    \item A clear justification of \underline{why} you have chosen this particular approach.
    \item Information needed for a reader to understand \underline{how} you did it (can a reader \underline{reproduce} your work, and collect equally valid results? e.g. hardware/software used, configuration, number of trials, any procedures used, etc.)
    \item A description of any approaches taken to process collected data, e.g. metrics are used to combine data in a meaningful way - you should state any used explicitly, their utility, their suitability to your methodology and their limitations.  
\end{itemize}



As on can see in Table \ref{tab:Table_with_numbers} there are numbers involved. 

%%%%%%%%%%%%%%%%%%%%%%%%%%%%%%%%%%%%
% If you have more complex tables you can 
% get a corresponding LaTeX code here
% https://www.tablesgenerator.com 
%%%%%%%%%%%%%%%%%%%%%%%%%%%%%%%%%%%%
\begin{table}[h!]
\centering
 \begin{tabular}{|c | c | c | c|} 
 \hline
  Frame number & User 1 state & User 2 state & Resulting state \\ [0.5ex] 
 \hline
 \hline
  n & 0 & 0 & 1 \\ 
 \hline
  n+1 & 0 & 1 & 2\\
 \hline
  n+2 & 1 & 0 & 3 \\
 \hline
  n+3 & 1 & 1 & 4 \\
 \hline
\end{tabular}
\caption{\label{tab:Table_with_numbers}An example of a table.}
\label{table:example}
\end{table}

For example, if $x>0$ then we can write
\begin{equation}
\label{eq:sum}
\sigma =\int_{x=0}^{\infty} \frac{1}{x^2}dx \quad ,
\end{equation}
where $\sigma$ is the integral (see Equations \ref{eq:sum}).  

}

This section provides detailed information on algorithm, its implementation, and experiment design.

\section{Algorithm and Implmentation}

\subsection*{Environment}

\begin{itemize}
    \item \textbf{Hardware:} ROG Zephyrus M16 Laptop
    \begin{itemize}
        \item CPU: 11th Gen Intel(R) Core(TM) 17-11800H @ 2.30GHz
        \item GPU: NVIDIA GeForce RTX 3060 Laptop GPU (unrelated to the experiment, information provided just for content completeness)
    \end{itemize}
    \item \textbf{Software:}
    \begin{itemize}
        \item OS: WSL2 (Ubuntu 22.04 LTS) in Windows 11 23H2
        \item Implementation Platform: ROS2 Humble, all codes written in python
    \end{itemize}
\end{itemize}


\textbf{The algorithm's idea} could be summarised in the following two parts: 
\begin{enumerate}
    \item \textbf{Communicate with STDMA}: Agents establish communication with STDMA protocol.
    \item \textbf{Plan}: Agents plan their move based on the information received.
\end{enumerate}

And they are explained as below.

\subsection{Communicate with STDMA}

As previously mentioned in Section \ref{sec:STDMA Explained}, the STDMA protocol mainly consists of the following three parts:
\begin{enumerate}
    \item Synchronised Clock: Agents have synchronised clock.
    \item State Machine: Four states for an agent to control their behaviour and describe network usage.
    \item Message Publishing and Sending: Agents could monitor the channel and send information through it.
\end{enumerate}

And these features are implemented within the ROS2 structure.

\subsubsection*{Synchronised Clock}

In this paper, shared clock pulses are used to achieve the purpose of synchronizing clocks between agents.
A ROS2 publisher sends a square wave with a fixed period and a duty cycle of 50$\%$ to the designated topic, serving as the shared clock signal. 
Agents subscribe to this clock signal topic to achieve clock synchronisation.





\section{Experiment Design}


