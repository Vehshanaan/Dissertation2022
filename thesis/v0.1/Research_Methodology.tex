
%%%%%%%%%%%%%%%%%%%%%%%%%%%%%%%%%%%%
\chapter{Algorithm}
\label{chap:Algorithm}
%%%%%%%%%%%%%%%%%%%%%%%%%%%%%%%%%%%%

\comment{

The Methodology section should provide a clear explanation of the research approach.  You want your reader to agree that you carefully considered your method so that we can trust your results to be both insightful (\underline{mean something}) and credible (\underline{not subject to error}):
\begin{itemize}
    \item A clear description of the methodology, how it creates a scientific investigation and operates to collect meaningful data.
    \item A clear justification of \underline{why} you have chosen this particular approach.
    \item Information needed for a reader to understand \underline{how} you did it (can a reader \underline{reproduce} your work, and collect equally valid results? e.g. hardware/software used, configuration, number of trials, any procedures used, etc.)
    \item A description of any approaches taken to process collected data, e.g. metrics are used to combine data in a meaningful way - you should state any used explicitly, their utility, their suitability to your methodology and their limitations.  
\end{itemize}



As on can see in Table \ref{tab:Table_with_numbers} there are numbers involved. 

%%%%%%%%%%%%%%%%%%%%%%%%%%%%%%%%%%%%
% If you have more complex tables you can 
% get a corresponding LaTeX code here
% https://www.tablesgenerator.com 
%%%%%%%%%%%%%%%%%%%%%%%%%%%%%%%%%%%%
\begin{table}[h!]
\centering
 \begin{tabular}{|c | c | c | c|} 
 \hline
  Frame number & User 1 state & User 2 state & Resulting state \\ [0.5ex] 
 \hline
 \hline
  n & 0 & 0 & 1 \\ 
 \hline
  n+1 & 0 & 1 & 2\\
 \hline
  n+2 & 1 & 0 & 3 \\
 \hline
  n+3 & 1 & 1 & 4 \\
 \hline
\end{tabular}
\caption{\label{tab:Table_with_numbers}An example of a table.}
\label{table:example}
\end{table}

For example, if $x>0$ then we can write
\begin{equation}
\label{eq:sum}
\sigma =\int_{x=0}^{\infty} \frac{1}{x^2}dx \quad ,
\end{equation}
where $\sigma$ is the integral (see Equations \ref{eq:sum}).  

}

This section provides detailed information on algorithm and its implementation.

\section*{Environment}

\begin{itemize}
    \item \textbf{Hardware:} ROG Zephyrus M16 Laptop
    \begin{itemize}
        \item CPU: 11th Gen Intel(R) Core(TM) 17-11800H @ 2.30GHz
        \item GPU: NVIDIA GeForce RTX 3060 Laptop GPU (unrelated to the experiment, information provided just for content completeness)
    \end{itemize}
    \item \textbf{Software:}
    \begin{itemize}
        \item OS: WSL2 (Ubuntu 22.04 LTS) in Windows 11 23H2
        \item Implementation Platform: ROS2 Humble, all codes written in python
    \end{itemize}
\end{itemize}

\section{Communication with STDMA}

In STDMA, agents share a single channel by autonomously determining the serial speaking order.
The method for determining the speaking order involves agents autonomously allocating the right to use free times within the channel.


\subsection{Synchronised Clock}

% STDMA假设agent之间拥有同步时钟。
\begin{quotation}
    \textbf{Assumption 1}: 
    STDMA assumes that agents have synchronised clocks. 
\end{quotation}

% 在实际中,同步时钟一般用GPS实现。在本文的模拟中,用一个ROS2publisher和一个topic来实现。
In practice, the synchronised clock is typically implemented with GPS. 
In the implemented simulation of this paper, it's achieved using a ROS2 publisher and a topic.
% 一个专用的ros2节点定时翻转其成员bool值,并在每次翻转此值时将翻转后的bool值publish到时钟topic中,这样在时钟topic中来形成一个占空比为50%的方波时钟信号。
A dedicated ROS2 node periodically toggles its member boolean value and publishes the toggled value to the clock topic each time it's flipped. This creates a square wave clock signal with a 50\% duty cycle in the clock topic.
% 其他普通agent通过订阅时钟话题的方式来获得同步时钟信号。
Other standard agents obtain the synchronised clock signal by subscribing to the clock topic.

\subsection{State Machine for Channel Allocation}

